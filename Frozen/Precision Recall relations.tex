\documentclass{article}
\usepackage[dvipsnames]{xcolor}
\usepackage{tikz}
\usepackage{amsmath}
\usepackage{amssymb}
\usepackage{nicefrac}
%\usepackage{mathtools}

\makeatletter

\newcommand*{\twohalfcircle}[2]{%
  \ensuremath{%
    \mathord{% change accordingly to the symbol type
      \mathpalette{\@twohalfcircle{#1}{#2}}{}%
    }%
  }%
}

\newcommand*{\colorcircle}[1]{%
  \ensuremath{%
    \mathord{% change accordingly to the symbol type
      \mathpalette{\@colorcircle{#1}}{}%
    }%
  }%
}

\newdimen\@twohalfcircle@dimen
\newdimen\@twohalfcircle@linewidth
\newdimen\@twohalfcircle@radius
\newdimen\@colorcircle@dimen
\newdimen\@colorcircle@linewidth
\newdimen\@colorcircle@radius

\newcommand*{\@twohalfcircle}[4]{%
  % #1: bottom color
  % #2: top color
  % #3: math style
  % #4: unused
  \sbox0{$#3A$}%
  \@twohalfcircle@dimen=\ht0 %
  \@twohalfcircle@linewidth=.06\@twohalfcircle@dimen
  \pgfmathsetlength\@twohalfcircle@radius{%
    .5\@twohalfcircle@dimen
    - .5\@twohalfcircle@linewidth
  }%
  \begin{tikzpicture}[
    baseline=-.5\@twohalfcircle@dimen,
    x=\@twohalfcircle@dimen,
    y=\@twohalfcircle@dimen,
    radius=\@twohalfcircle@radius,
    line width=\@twohalfcircle@linewidth,
  ]
    \filldraw[fill={#1}] circle[];
    \filldraw[fill={#2}]
      (-\@twohalfcircle@radius, 0)
      arc(180:360:\@twohalfcircle@radius)
      -- cycle
    ;
  \end{tikzpicture}
}

\newcommand*{\@colorcircle}[3]{%
  % #1: color
  % #3: math style
  % #4: unused
  \sbox0{$#2A$}%
  \@colorcircle@dimen=\ht0 %
  \@colorcircle@linewidth=.06\@colorcircle@dimen
  \pgfmathsetlength\@colorcircle@radius{%
    .5\@colorcircle@dimen
    - .5\@colorcircle@linewidth
  }%
  \begin{tikzpicture}[
    baseline=-.5\@colorcircle@dimen,
    x=\@colorcircle@dimen,
    y=\@colorcircle@dimen,
    radius=\@colorcircle@radius,
    line width=\@colorcircle@linewidth,
  ]
    \filldraw[fill={#1}] circle[];
  \end{tikzpicture}
}

\newcommand{\colsupcol}[2]{\colorcircle{#1}^{\colorcircle{#2}}}

\makeatother

\def \poscol {YellowOrange}
\def \negcol {gray}
\def \allcol {white}

%\DeclareMathSizes{10}{40}{40}{40}

\newcommand{\poss}{\twohalfcircle{\poscol}{\allcol}}
\newcommand{\negs}{\twohalfcircle{\negcol}{\allcol}}
\newcommand{\fallscore}{\phi}

\begin{document}
The format is $\text{Name}=\text{True}^{\text{Pred}}$ with $\text{Positive}=\colorcircle{\poscol}$, $\text{Negative}=\colorcircle{\negcol}$ and $\text{All}=\colorcircle{\allcol}$.

\addtolength{\jot}{1em}
\begin{align*}
\text{Precision}&=\frac{\colsupcol{\poscol}{\poscol}}{\colsupcol{\allcol}{\poscol}}=\text{Recall}\cdot\frac{\colorcircle{\poscol}}{\colsupcol{\allcol}{\poscol}}\\
\text{Recall}&=\frac{\colsupcol{\poscol}{\poscol}}{\colorcircle{\poscol}}=\text{TPR}\\
\text{Fall-out}&=\frac{\colsupcol{\negcol}{\poscol}}{\colorcircle{\negcol}}=\text{FPR}\\
\frac{1}{\text{Prec}}&=1+\frac{\colsupcol{\negcol}{\poscol}}{\colsupcol{\poscol}{\poscol}}=1+\frac{\text{Fall-out}}{\text{Recall}}\cdot\frac{\negs}{\poss}\\
\text{Accuracy}&=\frac{ \colsupcol{\poscol}{\poscol} + \colsupcol{\negcol}{\negcol}}{\colorcircle{\allcol}}=\poss\cdot\text{Recall}+\negs\cdot(1-\text{Fall-out})\\
&=\negs+\poss\cdot\text{Recall}\cdot\left(2-\frac{1}{\text{Prec}}\right)=1-2\cdot\poss\cdot\text{Recall}\cdot\left(\frac{1}{F_1}-1\right)\\
&=\left<\text{Recall}_i\right>_{\text{true}}=\left<\text{Prec}_i\right>_{\text{Pred}}\\
1&=\sum\frac{\poss_i}{\text{Prec}_i}=\sum \frac{\poss_i}{\text{Recall}_i}\\
\frac{1}{\poss}&=1+\frac{\colorcircle{\negcol}}{\colorcircle{\poscol}}\\
\text{CorrectPositives}&=\frac{\colsupcol{\poscol}{\poscol}}{\colorcircle{\allcol}}=\text{Recall}\cdot\poss
\end{align*}
\begin{align*}
\text{ROC-curve}&=\nicefrac{\colorcircle{\poscol}^{\twohalfcircle{\poscol}{\allcol}}}{\colorcircle{\negcol}^{\twohalfcircle{\poscol}{\allcol}}}=\nicefrac{\text{Recall}}{\text{Fall-out}}=\nicefrac{\text{TPR}}{\text{FPR}}\\
\text{Matthews corr.coef.}&=\frac{\colsupcol{\poscol}{\poscol}\cdot\colsupcol{\negcol}{\negcol}-\colsupcol{\negcol}{\poscol}\cdot\colsupcol{\poscol}{\negcol}}{\sqrt{\colsupcol{\allcol}{\poscol}\cdot\colsupcol{\allcol}{\negcol}\cdot\colorcircle{\poscol}\cdot\colorcircle{\negcol}}}=\frac{\text{CP}\cdot\text{Prec}\cdot\text{Recall}-1}{\sqrt{(\text{CP}\cdot\text{Prec}-1)(\text{CP}\cdot\text{Recall}-1)}}\\
\frac{1+\beta^2}{F_\beta}&=\frac{1}{\text{Prec}}+\frac{\beta^2}{\text{Recall}}\\
F_\beta&=\frac{(1+\beta^2)\colsupcol{\poscol}{\poscol}}{(1+\beta^2)\colsupcol{\poscol}{\poscol}+\beta^2\colsupcol{\poscol}{\negcol}+\colsupcol{\negcol}{\poscol}}\\
\text{AUC}&\geq\frac{\text{Recall}-\text{Fallout}+1}{2}=\text{Score}\left(\phi=\frac{\colorcircle{\poscol}}{\colorcircle{\negcol}}\right)\\
\frac{1}{\text{Prec}}&\geq 2\cdot (1-\text{AUC})\cdot\frac{\poss}{\negs}+1\\
\frac{1}{\text{Prec}}&\gtrsim \text{AUC}'\cdot\frac{\poss}{\negs}\qquad(\text{symmetric curve})\\
2\cdot\text{AUC}-1&=\text{Recall}-\text{Fallout}=\text{Recall}\cdot\left(1+\frac{\poss}{\negs}\left(1-\frac{1}{\text{Prec}}\right)\right)\\
\text{Recall}-\text{Fallout}&=\left(\colorcircle{\poscol}^\alpha-\colorcircle{\allcol}^\alpha\poss\right)\cdot\frac{\colorcircle{\allcol}}{\colorcircle{\poscol}\colorcircle{\negcol}}\\
\colsupcol{\allcol}{\poscol}&=\frac{\text{Recall}}{\text{Prec}}\cdot \colorcircle{\poscol}\\
\text{Mutual Info}&=\psi(\poss\cdot \text{Recall}+\negs\cdot\text{Fallout})-\poss\cdot\psi(\text{Recall})-\negs\cdot\psi(\text{Fallout})\\
\psi(x)&=-x\log(x)-(1-x)\log(1-x)\\
\psi'(x)&=\log\frac{1-x}{x}
\end{align*}

For plots $x=p-\alpha n$ ($p$ number real positive until cutoff, $n$ number total until cutoff) and $y=p-\beta n$, you have that plot slopes $\Delta y/\Delta x$ are always a function of precision $f(\Delta p/\Delta n)$. All $p, n$ can additionally also have factor $0$ or $-1$.

Constant $F_\beta$ lines change height and slope with different $F_\beta$ and hence are less useful. The last equation comes from an trapezoid approximation in the ROC curve.

Since number of true positives and negatives are fixed, you can set one score within the true classes to zero (substract minimum cost in true class). Effectively we are left with only one parameter for binary classification costs:
\begin{align*}
S(\colsupcol{\poscol}{\poscol})&=1 & S(\colsupcol{\poscol}{\negcol})&=0\\
S(\colsupcol{\negcol}{\poscol})&=-\fallscore & S(\colsupcol{\negcol}{\poscol})&=0
\end{align*}
$\fallscore=1$ would stand for plain Accuracy scoring ($\text{Accuracy}=\text{Score}_{\fallscore=1}\cdot\poss+\negs$).
\begin{align*}
\text{Score}&=\frac{\sum S}{\colorcircle{\poscol}}=\frac{\colsupcol{\poscol}{\poscol}-\phi\cdot\colsupcol{\negcol}{\poscol}}{\colorcircle{\poscol}}\\
&=\text{Recall}-\text{Fall-out}\cdot\fallscore\cdot\frac{\negs}{\poss}\\
&=\text{Recall}\cdot\left(1+\phi-\frac{\phi}{\text{Prec}}\right)\\
1+\fallscore&=\frac{\text{Score}}{\text{Recall}}+\frac{\fallscore}{\text{Prec}}\\
\text{Recall}&=\text{Fall-out}\cdot \fallscore\cdot\frac{\negs}{\poss}+\text{Score}\\
\text{Recall}&=\text{Fall-out}\cdot\frac{\negs}{\poss}\cdot\frac{\text{Prec}}{1-\text{Prec}}\\
\text{Recall}&=\frac{F_\beta}{1+\beta^2-F_\beta}\left(\text{Fall-out}\cdot\frac{\negs}{\poss}+\beta^2\right)\\
\text{Recall}&=-\text{Fall-out}\cdot \frac{\negs}{\poss}+\frac{1}{\colorcircle{\poscol}}\cdot \colsupcol{\allcol}{\poscol}
\end{align*}
The following are linear in the ROC-plot
\begin{itemize}
\item Constant cost matrix (parallel slopes)
\item Constant accuracy (parallel slopes, like cost matrix)
\item Constant precision (all through origin)
\item Constant $F_\beta$ score ($\beta$ determines pivot point)
\end{itemize}

For over-/undersampling the Recall and Fall-out (and hence ROC curve) stay exactly the same (assuming the predictive power is homogeneous across the instances). Then the transformed real positive label rate can be found from:
\begin{align*}
\colorcircle{\allcol}^{\poss}&=\text{Recall}\cdot\poss+\text{Fall-out}\cdot\negs\\
\text{Recall}\cdot(1+\fallscore)&=\colorcircle{\allcol}^{\poss}\cdot\frac{\fallscore}{\poss}+\text{Score}\\
\text{Recall}&=\colorcircle{\allcol}^{\poss}\cdot\frac{\text{Prec}}{\poss}
\end{align*}


\end{document}
